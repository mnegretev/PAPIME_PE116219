\documentclass[letterpaper,12pt]{article}
\usepackage[spanish]{babel}
\spanishdecimal{.}
\usepackage[utf8]{inputenc}
\usepackage{graphicx}
\usepackage[top=2.5cm, bottom=2.5cm, left=2.5cm, right=2.5cm]{geometry}
\usepackage{hyperref}

\title{Práctica 6 \\ Generación de Trayectorias de Caminado}
\author{Robots Bípedos Autónomos}
\date{Facultad de Ingeniería, UNAM}


\begin{document}
\renewcommand{\tablename}{Tabla}
\maketitle
\section*{Objetivos}
\begin{itemize}
\item Introducir las diferentes técnicas para generación de trayectorias de caminado.
\item Escribir un programa que genere trayectorias para ambas piernas usando alguno de los métodos vistos.
\item Probar las trayectorias tanto en el simulador como en el robot real.  
\end{itemize}

\section{Introducción}

\section{Desarrollo}

\section{Evaluación}
\end{document}