\documentclass[letterpaper,12pt]{article}
\usepackage[spanish]{babel}
\spanishdecimal{.}
\usepackage[utf8]{inputenc}
\usepackage{graphicx}
\usepackage[top=2.5cm, bottom=2.5cm, left=2.5cm, right=2.5cm]{geometry}
\usepackage{hyperref}
\usepackage{amssymb}

\title{Práctica 3 \\ Transformaciones Homogéneas}
\author{Robots Bípedos Autónomos}
\date{Facultad de Ingeniería, UNAM}


\begin{document}
\renewcommand{\tablename}{Tabla}
\maketitle
\section*{Objetivos}
\begin{itemize}
\item Familiarizar al alumno con el concepto de Transformación Homogénea.
\item Aprender a utilizar el paquete \texttt{tf} para el manejo de transformaciones homogéneas.
\item Familiarizar al alumno con los archivos \texttt{urdf} para descripción de cadenas cinemáticas.
\end{itemize}

\section{Introducción}
\subsection{Movimiento rígido}
Para entender el movimiento en los robots bípedos, es necesario comprender primero los conceptos de movimiento rígido y transformación homogénea. Un movimiento rígido es una combinación de una posición y una orientación, es decir es un par ordenado $(d,R)$, donde $d\in \mathbb{R}^3$ y $R\in SO(3)$.

El conjunto $SO(3)$ se refiere al conjunto de matrices ortogonales de orden 3 (del inglés \textit{Special Orthogonal}). Las matrices del conjunto $SO(n)$ tienen varias propiedades especiales como el hecho de que su inversa es igual a su transpuesta, todos sus renglones y columnas son de magnitud 1 y ortogonales entre sí y que su determinante es siempre 1 (esto último si estas representan rotaciones de sistemas dextrógiros).

Entonces, un movimiento rígido es una combinación de una posición y una orientación. Estos movimientos también pueden ser usados para representar la Transformación de un sistema coordenado a otro, es decir, el par $(d, R)$ puede representar una rotación y una traslación. Es importante mencionar que, aunque la rotación $R$ puede darse sobre cualquier eje, en general siempre se manejan rotaciones sobre alguno de los ejes coordenados, por lo que $R$, en general, tiene alguna de las siguientes formas:
\[R_{x,\theta} = \left[
    \begin{tabular}{ccc}
      1 & 0 & 0\\
      0 & $\cos \theta$ & $-\sin \theta$\\
      0 & $\sin \theta$ & $\cos \theta$
    \end{tabular}
  \right]\qquad
  R_{y,\theta} = \left[
    \begin{tabular}{ccc}
      $\cos\theta$ & 0 & $\sin\theta$\\
      0 & 1 & 0\\
      $-\sin\theta$ & 0 & $\cos\theta$
    \end{tabular}
  \right]
\]
\[R_{z,\theta} = \left[
    \begin{tabular}{ccc}
      $\cos \theta$ & $-\sin \theta$ & 0\\
      $\sin \theta$ & $\cos \theta$  & 0\\
      0 & 0 & 1
    \end{tabular}
  \right]\]

\subsection{Transformaciones homogéneas}
\subsection{El paquete \textit{tf}}
\subsection{El formato \textit{urdf}}

\section{Desarrollo}

\section{Evaluación}

\end{document}
