\documentclass[letterpaper,12pt]{article}
\usepackage[spanish]{babel}
\spanishdecimal{.}
\usepackage[utf8]{inputenc}
\usepackage{graphicx}
\usepackage[top=2.5cm, bottom=2.5cm, left=2.5cm, right=2.5cm]{geometry}
\usepackage{hyperref}

\title{Práctica 4 \\ Cálculo de Cinemática Directa}
\author{Robots Bípedos Autónomos}
\date{Facultad de Ingeniería, UNAM}


\begin{document}
\renewcommand{\tablename}{Tabla}
\maketitle
\section*{Objetivos}
\begin{itemize}
\item Comprender el concepto de cinemática directa.
\item Utilizar el paquete \texttt{tf} para el cálculo de la cinemática directa.
\item Calcular la posición del tronco del robot con respecto al centro de la planta de uno de los pies.
\end{itemize}

\section{Introducción}

\section{Desarrollo}

\section{Evaluación}

\end{document}